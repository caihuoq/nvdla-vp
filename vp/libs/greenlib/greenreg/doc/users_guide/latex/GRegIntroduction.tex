% (c) GreenSocs Ltd
% author: Christian Schroeder <schroeder@eis.cs.tu-bs.de>

\cleardoublepage
\chapter{Introduction}

The \GreenReg project\footnote{\GreenReg project page:  \href{http://www.greensocs.com/projects/GreenReg}{http://www.greensocs.com/projects/GreenReg}} provides a library that can be used for device and register modeling in ESL design. The project is based on the Intel \DRF framework.

\GreenReg can be understood as two parts: One side is the device and register modeling in the user model which is done by using the Device Register Framework (\DRF) API, the other is the configuration mechanism which is provided by \GreenConfig\footnote{\GreenConfig project page:  \href{http://www.greensocs.com/projects/GreenControl/GreenConfig}{http://www.greensocs.com/projects/GreenControl/GreenConfig}} and gives configuration tool access to the \DRF objects.

This allows ESL modelers to use the powerful configuration abilities of \GreenConfig to configure the \DRF objects.

The \GreenReg framework is not only a register framework. The way the user is able to model hardware registers and use their notifications for modeling behavior is a small model of computation.

%%%%%%%%%%%%%%%%%%%%%%%%%%%%%%%
\section{Further Reading}

This User's Guide focuses on the configuration aspect of \GreenReg and some enhancements to DRF.

See the \GreenReg Tutorial Slides\footnote{\hypertarget{lnk:TutorialSlides}{\GreenReg Tutorial Slides}:  \href{http://www.greensocs.com/Projects/GreenReg/docs/GreenRegTutorial}{http://www.greensocs.com/Projects/GreenReg/docs/GreenRegTutorial}} for more basic usage information and an introduction to the device and register modeling part. 
